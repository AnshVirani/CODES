\documentclass{article}
\newcounter{rownumbers}
\newcommand\rownumber{\stepcounter{rownumbers}\arabic{rownumbers}}
%\usepackage[a4paper, total={6in, 8in}]{geometry}
\usepackage{graphicx}
\usepackage{bm}
\usepackage{amsmath,amsfonts,mathtools}
\usepackage{amsthm}
\newtheorem{theorem}{Theorem}[section]
\newtheorem{lemma}[theorem]{Lemma}

\graphicspath{ {E:\College\Sem-6\PSP\CODES\TeX\Assignment-3} }
\usepackage{geometry}
 \geometry{
 a4paper,
 total={210mm,297mm},
 left=20mm,
 right=20mm,
 top=-2mm,
 bottom=2mm,
 }
 
%\usepackage[margin=0.5in]{geometry}

\usepackage{amsmath,amssymb}
\usepackage{ifpdf}
%\usepackage{cite}
\usepackage{algorithmic}
\usepackage{array}
\usepackage{mdwmath}
\usepackage{pdfpages}
\usepackage{mdwtab}
\usepackage{eqparbox}
\usepackage{parskip}
%\onecolumn
%\input{psfig}
\usepackage{color}
\usepackage{graphicx}
\setlength{\textheight}{23.5cm} \setlength{\topmargin}{-1.05cm}
\setlength{\textwidth}{6.5in} \setlength{\oddsidemargin}{-0.5cm}
\renewcommand{\baselinestretch}{1}
\pagenumbering{arabic}
\linespread{1.15}
\begin{document}
\textbf{
\begin{center}
{
\large{School of Engineering and Applied Science (SEAS), Ahmedabad University}\vspace{5mm}
}
\end{center}
%
\begin{center}
\large{Probability and Stochastic Processes (MAT277)\\ \vspace{4mm}
Homework Assignment-4\\\vspace{2mm}
Enrollment No: AU2140096 \hspace{4cm} Name: Ansh Virani }
\end{center}}
\vspace{2mm}


\vspace{10mm}

\begin{enumerate}
\item 

\newpage
\item 
    A random variable \(X\) is uniformly distributed over the interval (0, 1) and related to \(Y\) by,
    \[
        \tan\left(\cfrac{\pi Y}{2}\right) = e^X \implies Y = \cfrac{2}{\pi}\arctan(e^X)
    \]
    \[
        \therefore\ \ \cfrac{dY}{dX} = \cfrac{2}{\pi}\cdot\cfrac{1}{1+e^{2X}}
    \]
    Applying the the transformation rule, we get:
    \[
        f_Y(y) = f_X(x)\left|\cfrac{dY}{dX}\right| = 1 \times \cfrac{2}{\pi}\cdot\cfrac{1}{1+e^{2X}}
    \]
    Since X is expressed in terms of Y through the initial transformation, $e^X = \tan\left(\cfrac{\pi Y}{2}\right)$, the \(PDF\) can be expressed in terms of Y as follows:
    \[
        f_Y(y) = \left(\cfrac{2}{\pi}\right)\left(\cfrac{1}{1+\tan^2\left(\cfrac{\pi y}{2}\right)}\right)
    \]
    Using the identity $1 + \tan^2(z) = \sec^2(z)$, we get:
    \[
        f_Y(y) = \left(\cfrac{2}{\pi}\right)\left(\cfrac{1}{sec^2\left(\cfrac{\pi y}{2}\right)}\right) = \cfrac{2}{\pi}\cos^2\left(\cfrac{\pi y}{2}\right)
    \]\\
    By solving for \(Y\), computing the derivative with respect to \(X\), and applying the transformation rule, the resulting \(PDF\) for \(Y\) is $f_Y(y) = \cfrac{2}{\pi}\cos^2\left(\cfrac{\pi y}{2}\right)$, valid for \(y\) in the interval $(0, 1)$.
    
    

\end{enumerate}
\end{document}