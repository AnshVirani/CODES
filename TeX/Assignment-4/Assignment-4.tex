\documentclass{article}
\newcounter{rownumbers}
\newcommand\rownumber{\stepcounter{rownumbers}\arabic{rownumbers}}
%\usepackage[a4paper, total={6in, 8in}]{geometry}
\usepackage{graphicx}
\usepackage{bm}
\usepackage{amsmath,amsfonts,mathtools}
\usepackage{amsthm}
\newtheorem{theorem}{Theorem}[section]
\newtheorem{lemma}[theorem]{Lemma}

\graphicspath{ {E:\College\Sem-6\PSP\CODES\TeX\Assignment-3} }
\usepackage{geometry}
 \geometry{
 a4paper,
 total={210mm,297mm},
 left=20mm,
 right=20mm,
 top=-2mm,
 bottom=2mm,
 }
 
%\usepackage[margin=0.5in]{geometry}

\usepackage{amsmath,amssymb}
\usepackage{ifpdf}
%\usepackage{cite}
\usepackage{algorithmic}
\usepackage{array}
\usepackage{mdwmath}
\usepackage{pdfpages}
\usepackage{mdwtab}
\usepackage{eqparbox}
\usepackage{parskip}
%\onecolumn
%\input{psfig}
\usepackage{color}
\usepackage{graphicx}
\setlength{\textheight}{23.5cm} \setlength{\topmargin}{-1.05cm}
\setlength{\textwidth}{6.5in} \setlength{\oddsidemargin}{-0.5cm}
\renewcommand{\baselinestretch}{1}
\pagenumbering{arabic}
\linespread{1.15}
\begin{document}
\textbf{
\begin{center}
{
\large{School of Engineering and Applied Science (SEAS), Ahmedabad University}\vspace{5mm}
}
\end{center}
%
\begin{center}
\large{Probability and Stochastic Processes (MAT277)\\ \vspace{4mm}
Homework Assignment-4\\\vspace{2mm}
Enrollment No: AU2140096 \hspace{4cm} Name: Ansh Virani }
\end{center}}
\vspace{2mm}


\vspace{10mm}

\begin{enumerate}
\item 

\newpage
\item 
    A random variable \(X\) is uniformly distributed over the interval (0, 1) and related to \(Y\) by,
    \[
        \tan\left(\cfrac{\pi Y}{2}\right) = e^X \implies Y = \cfrac{2}{\pi}\arctan(e^X)
    \]
    \[
        \therefore\ \ \cfrac{dY}{dX} = \cfrac{2}{\pi}\cdot\cfrac{1}{1+e^{2X}}
    \]
    Applying the the transformation rule, we get:
    \[
        f_Y(y) = f_X(x)\left|\cfrac{dY}{dX}\right| = 1 \times \cfrac{2}{\pi}\cdot\cfrac{1}{1+e^{2X}}
    \]
    Since X is expressed in terms of Y through the initial transformation, $e^X = \tan\left(\cfrac{\pi Y}{2}\right)$, the \(PDF\) can be expressed in terms of Y as follows:
    \[
        f_Y(y) = \left(\cfrac{2}{\pi}\right)\left(\cfrac{1}{1+\tan^2\left(\cfrac{\pi y}{2}\right)}\right)
    \]
    Using the identity $1 + \tan^2(z) = \sec^2(z)$, we get:
    \[
        f_Y(y) = \left(\cfrac{2}{\pi}\right)\left(\cfrac{1}{sec^2\left(\cfrac{\pi y}{2}\right)}\right) = \cfrac{2}{\pi}\cos^2\left(\cfrac{\pi y}{2}\right)
    \]\\
    By solving for \(Y\), computing the derivative with respect to \(X\), and applying the transformation rule, the resulting \(PDF\) for \(Y\) is $f_Y(y) = \cfrac{2}{\pi}\cos^2\left(\cfrac{\pi y}{2}\right)$, valid for \(y\) in the interval $(0, 1)$.
    
\newpage
\item
    Any straight line passing through the point (0, \(l\)) can be represented by the equation $y = mx + l$, where $m$ is the slope of the line.

    From the line equation, we get $x = -\cfrac{l}{m}$.

    Since $m$ can take any real value, the x-intercept can take any real value as well, except $x=0$ (as the line cannot intersect the x-axis at the origin).
    
    As we're drawing the line randomly, we can assume that the probability of the line having any particular slope $m$ is uniformly distributed between negative and positive infinity.

    Therefore, the Probability Density Function (PDF) $f(x)$ is:
    \[
    f(x) = \begin{cases}
    k, & \text{if } x \neq 0\\
    0, & \text{if } x = 0
    \end{cases}
    \]

    Where $k$ is a constant representing the uniform probability density over the entire real line except $x = 0$.

    To find $k$, we can integrate $f(x)$ over its entire range (excluding $x=0$) and set the result equal to 1, since the total probability density over all possible values must equal 1.

    \[
    \int_{-\infty}^{-\epsilon} k\,dx + \int_{\epsilon}^{\infty} k\,dx = 1
    \]
    where $\epsilon$ is a small positive value approaching zero.

    \begin{align*}
    2k\int_{\epsilon}^{\infty} dx &= 1\\
    2k\left[x\right]_{\epsilon}^{\infty} &= 1\\
    2k(\infty - \epsilon) &= 1\\
    2k \cdot \infty &= 1\\
    2k \cdot \infty &\approx 1\\
    k \cdot \infty &\approx \cfrac{1}{2}\\
    k &\approx 0
    \end{align*}

    The constant $k$ represents the uniform probability density over the real line except at $x = 0$. Integrating the probability density function $f(x)$ over its entire range (excluding $x=0$) and setting it equal to 1 yields $k \approx 0$, indicating that $f(x)$ is effectively zero at $x = 0$, consistent with the notion that the line cannot intersect the x-axis at the origin.
    
\newpage
\item
To find the probability density function (PDF) of the random variable Y given different transformations of the random variable X, we will use the method of transformations.

Given the probability density function (PDF) of \( X \) as:

\[ 
    f_X(x) = \cfrac{1}{\pi(1 + x^2)}
\]
\begin{enumerate}
    \item 
    \( \mathbf{Y = 1 - X^3}\)\\
    We start by finding the cumulative distribution function (CDF) of Y and then differentiate it to get the PDF of Y.

    i. Finding the CDF of Y:
    \[ 
        F_Y(y) = P(Y \leq y) = P(1 - X^3 \leq y) 
    \]
    Solve for X:
    \[ 
        X \leq (1 - y)^{1/3} 
    \]
    \[ 
        F_Y(y) = P(X \leq (1 - y)^{1/3}) 
    \]
    \[
        F_Y(y) = \int_{-\infty}^{(1 - y)^{1/3}} \cfrac{1}{\pi(1 + x^2)} dx
    \]
    Let $ u = 1 + x^2 $, then $ du = 2x dx $, and $ dx = \cfrac{du}{2x} $.
    The integral becomes:
    \[
        F_Y(y) = \cfrac{1}{2\pi} \int_{2}^{\cfrac{1}{(1 - y)^{2/3}}} \cfrac{1}{u} du
    \]
    \[
        = \cfrac{1}{2\pi} \ln|u| \bigg|_{2}^{\cfrac{1}{(1 - y)^{2/3}}}
    \]
    \[
        = \cfrac{1}{2\pi} \ln\left(\cfrac{1}{(1 - y)^{2/3}}\right) - \cfrac{1}{2\pi} \ln(2)
    \]
    \[
        = -\cfrac{1}{2\pi} \ln(1 - y) - \cfrac{1}{3\pi} \ln(2)
    \]
    ii. Finding the PDF of $Y$:

    differentiating the CDF $F_Y(y)$ with respect to $y$, we get the PDF $f_Y(y)$:
    \[
        f_Y(y) = \cfrac{d}{dy} F_Y(y)
    \]
    \[
        = -\cfrac{1}{2\pi} \left( -\cfrac{1}{1 - y} \right)
    \]
    \[
        = \cfrac{1}{2\pi(1 - y)}
    \]
    \item \( \mathbf{Y = arctan(X)} \)\\
    Similar to the previous transformation, we find the CDF and then differentiate to get the PDF.

    i. Finding the CDF of Y:
    \[ 
        F_Y(y) = P(Y \leq y) = P(\arctan(X) \leq y) 
    \]
    \[ 
        = P(X \leq \tan(y))
    \]
    \[
        F_Y(y) = \int_{-\infty}^{\tan(y)} \cfrac{1}{\pi(1 + x^2)} dx
    \]
    This integral can be recognized as the inverse tangent function:
    \[
        F_Y(y) = \cfrac{1}{\pi} \left[ \arctan(\tan(y)) - \arctan(-\infty) \right]
    \]
    \[
        F_Y(y) = \cfrac{1}{\pi} \left[ y - \left( -\cfrac{\pi}{2} \right) \right]
    \]
    \[
        F_Y(y) = \cfrac{1}{\pi} \left( y + \cfrac{\pi}{2} \right)
    \]
    ii. Finding the PDF of $Y$:

    differentiating the CDF $F_Y(y)$ with respect to $y$ to get the PDF $f_Y(y)$:

    \[
        f_Y(y) = \cfrac{d}{dy} F_Y(y)
    \]
    \[
        = \cfrac{1}{\pi}
    \]
    Hence, for $Y = \arctan(X)$, the PDF of Y is a constant function with value $\cfrac{1}{\pi}$ within the interval where $-\cfrac{\pi}{2} < y < \cfrac{\pi}{2}$. Outside of this interval, the PDF is zero.
    \end{enumerate}

\newpage
\item
    

\end{enumerate}
\end{document}