\documentclass{article}
\newcounter{rownumbers}
\newcommand\rownumber{\stepcounter{rownumbers}\arabic{rownumbers}}
%\usepackage[a4paper, total={6in, 8in}]{geometry}
\usepackage{graphicx}
\usepackage{bm}
\usepackage{amsmath,amsfonts,mathtools}

\graphicspath{ {C:\Users\harsh\OneDrive\Documents} }
\usepackage{geometry}
 \geometry{
 a4paper,
 total={210mm,297mm},
 left=20mm,
 right=20mm,
 top=-2mm,
 bottom=2mm,
 }
 
%\usepackage[margin=0.5in]{geometry}

\usepackage{amsmath,amssymb}
\usepackage{ifpdf}
%\usepackage{cite}
\usepackage{algorithmic}
\usepackage{array}
\usepackage{mdwmath}
\usepackage{pdfpages}
\usepackage{mdwtab}
\usepackage{eqparbox}
\usepackage{parskip}
%\onecolumn
%\input{psfig}
\usepackage{color}
\usepackage{graphicx}
\setlength{\textheight}{23.5cm} \setlength{\topmargin}{-1.05cm}
\setlength{\textwidth}{6.5in} \setlength{\oddsidemargin}{-0.5cm}
\renewcommand{\baselinestretch}{1}
\pagenumbering{arabic}
\linespread{1.15}
\begin{document}
\textbf{
\begin{center}
{
\large{School of Engineering and Applied Science (SEAS), Ahmedabad University}\vspace{5mm}
}
\end{center}
%
\begin{center}
\large{Probability and Stochastic Processes (MAT277)\\ \vspace{4mm}
Homework Assignment-1\\\vspace{2mm}
Enrollment No: AU2140096 \hspace{4cm} Name: Ansh Virani }
\end{center}}
\vspace{2mm}


\vspace{2mm}

\begin{enumerate}
\item \textbf{While tossing a biased die, calculate the probability that
face 3 has turned up, Given Alex tells either face 3 or face 6 has turned up.}

\begin{enumerate}
  \item\textit{We are given that,}
        \begin{center}
        \begin{tabular}{ | m{5em} | m{1cm}| m{1cm} | m{1cm} | m{1cm} | m{1cm} | m{1cm} | }
        \hline
        Face & 1 & 2 & 3 & 4 & 5 & 6\\
        \hline
        Probability & 0.2 & 0.22 & 0.11 & 0.25 & 0.15 & 0.07\\
        \hline
        \end{tabular}
        \end{center}

        \textit{Let A be the event that face 3 has turned up and B be the event that face 6 has turned up.}\\\\
        $\therefore $ $\Pr(A) = 0.11\ \ \ \&\ \ $ $\Pr(B) = 0.07$\\\\
        \textit{We know that these two are mutually exclusive events, hence:}\\\\
        $\begin{aligned}
        \therefore \Pr(A \cup B) & = \Pr(A) + \Pr(B)\\
        & = 0.11 + 0.32 \\
        & = 0.43
        \end{aligned}$\\\\
        \textit{Clearly, here we have to find the conditional probability, $\Pr(A \ |\ A \cup B)$}

        \begin{align*}
        \Pr(A \ |\ A \cup B) & = \frac{\Pr(A \cap (A \cup B))}{\Pr(A \cup B)}\\\\
        & = \frac{\Pr(A)}{\Pr(A \cup B)}\\\\
        & = \frac{0.11}{0.43}\\\\
        & = 0.2558139535.
        \end{align*}

        \[
          \boxed{\therefore\ \Pr(A \ |\ A \cup B)\ \ \approx \ \ 0.2559}
        \]
        
        \textit{Hence, the probability that face 3 has turned up, given either face 3 or face 6 has turned up is approx 25.59\%}
\end{enumerate}
\newpage
\item \textbf{There exists two events E1 \& E2 such that $\Pr(E1 \ |\ E2) = 0.45$, $\Pr(E2 \ |\ E1) = 0.5$ and $\Pr(E1 \cup E2) = 0.4$}
  \begin{enumerate}
    \item\textit{Calculate $\Pr(E1 \cap E2$):}\\\\
    \textit{According to Conditional Probability theorem we can say,}\\\\
    $\begin{aligned}
      \Pr(E1 \ |\ E2)\Pr(E2) & = \Pr(E2 \ |\ E1)\Pr(E1)\\
      0.45P(E2) & = 0.5P(E1) \\
    \end{aligned}$\\\\
    \textit{Applying Basic Probaility theorem:}\\\\
    $\begin{aligned}
      \Pr(E1 \cup E2) & = \Pr(E1) + \Pr(E2) + \Pr(E1 \cap E2)\\
      0.4 & = 0.9P(E2) + \Pr(E2) + \Pr(E1 \cap E2)\\\\
    \end{aligned}$

    $\therefore \Pr(E1 \cap E2) = 1.9P(E2) - 0.4$ \hspace{8cm}......(1)\\\\
    \textit{Now, Substituting these equation in conditional probability equation:}\\\\
    $\begin{aligned}
      \Pr(E1 \ |\ E2) & = \frac{\Pr(E1 \cap E2)}{\Pr(E2)}\\\\
      0.45 & = \frac{1.9P(E2) - 0.4}{\Pr(E2)}\\
      0.4 & = 1.45P(E2)\\\\
      \therefore \Pr(E2) = 0.275
    \end{aligned}$\\\\
    \textit{Substituting value of $\Pr(E2)$ in equation 1, we get}\\\\
    $\therefore \Pr(E1) = 0.2475$\\\\
    $\therefore \Pr(E1 \cap E2) = 0.1225$\\
    \item \textit{ Comment on the dependency relation between event E1 and E2:}
    
    \textnormal{When event E2 occurs, there's a 45\% chance that event E1 will occur. On the other hand if event E1 occurs, there's a 50\% chance that event E2 will occur.}
    \textnormal{This shows that the events are unsymmetrically related.}
    
    \textnormal{Also upon calculating we can see that, there's some shared occurance happening between two of the given events,}
    \textnormal{as probability of them happening together was found to be 12.25\%.}
  \end{enumerate}

\newpage
\item\textbf{Given Probabilies are:}\\
  \begin{enumerate}
    \item\textbf{Let's denote the probability of a Red ball as:  $\Pr(R) = 0.45$}
    \item\textbf{Let's denote the probability of a Striped ball as:  $\Pr(S) = 0.3$}
    \item\textbf{Let's denote the probability of a Red ball with stripes as:  $\Pr(RS) = \Pr(R \cap S) = 0.2$}\\
  \end{enumerate}
  \textit{To find the probability that ball is striped given the ball picked is a Red one.}\\
  \begin{align*}
    \Pr(S \ |\ R) & = \frac{\Pr(R \cap S)}{\Pr(R)}\\\\
    & = \frac{0.2}{0.45}\\\\
    & = 0.4444444...
  \end{align*}\\
  \[
     \boxed{\therefore\ \Pr(S \ |\ R )\ \ \approx \ \ 0.4445}\\
  \]

  \textit{Hence, the probability that the ball is striped one given the ball in red ball is approx 44.45\%}\\\\

\newpage  
\item \textbf{Let A denote the event where number 8 is obtained while tossing a 8-sided unbiased dice}\\
$\therefore \Pr(A) = \ \cfrac{1}{8}$ = \ p\\\\
$\therefore \Pr(A') = 1 - p $  

\textnormal{X denotes the number of tosses required to get number 8 as an outcome.}\\
\begin{enumerate}
    \item \textnormal{The probability that X = 6:}\\
    \textit{we use the equation:}
    \begin{align*}
      \Pr(X) & = \Pr(A')^{(X-1)}.\Pr(A)\\\\
      \Pr(X_6)\  & =\  (1-p)^{5}.p\\
      & =\ (\cfrac{7}{8})^{5}.\cfrac{1}{8}\\
      & =\ (0.875)^{5}*(0.125)\\
      & =\ 0.0641136169
    \end{align*}
    \[
     \boxed{\therefore\ \Pr(X_6)\ \ \approx \ \ 0.06412}\\
    \]
    \item \textnormal{Conditional Probability that $X \leq 6$ given $X < 9$:}
    \begin{center}
    $\begin{aligned}
      \Pr(X \leq 6 \ | \ X < 9) & = \frac{\Pr(X \leq 6 \cap X < 9)}{\Pr(X < 9)}\\
     & = \frac{\Pr(X \leq 6)}{\Pr(X < 9)}\\
     & = \dfrac{ \sum_{i=1}^{6}((1 - p)^{(i - 1)}*p)}{ \sum_{j=1}^{8}((1 - p)^{(j - 1)}*p)}\\
     & = \dfrac{ \sum_{i=1}^{6}(\frac{7}{8}^{(i - 1)}\frac{1}{8})}{ \sum_{j=1}^{8}(\frac{7}{8}^{(j - 1)}\frac{1}{8})}\\
     \end{aligned}$
   \end{center}
   \textnormal{Using the formula of Sum of Geometric series} 
   \begin{equation}                    
    a *( \dfrac{1-r^n}{1-r} ) \ where  \ a =\frac{1}{8}, \ r = \frac{7}{8}
    \end{equation}
   \textnormal{upon solving, we get} \begin{center}
       $\sum_{i=1}^{6}(\frac{7}{8}^{(i - 1)})$ and $\sum_{j=1}^{8}(\frac{7}{8}^{(j - 1)})$
   \end{center}
   \begin{center}
       $\begin{aligned}
          \dfrac{\sum_{i=1}^{6}(\frac{7}{8}^{(i - 1)}* \frac{1}{8})}{\sum_{j=1}^{8}(\frac{7}{8}^{(j - 1)}* \frac{1}{8})} = \frac{0.551204}{0.656391}\\
       \end{aligned}$
   \end{center}
   \begin{center}
       $\boxed{\therefore\ \Pr(X\leq 6 \ | \ X < 9)\ \ =  \ 0.8397\ }$
   \end{center}
\textit{Hence, the probability that $X \leq 6$ given $X < 9$ is $0.8397$ .}\\\\
  \end{enumerate}

\newpage
\item\textbf{Given Probabilies are:}\\
  \begin{enumerate}
    \item\textbf{probability that an employee arrives late:  $\Pr(A_l) = 0.15$}
    \item\textbf{probability that an employee leaves early:  $\Pr(L_e) = 0.25$}
    \item\textbf{probability that an employee arrives late and leaves early:  $\Pr(A_l \cap L_e) = 0.08$}\\
  \end{enumerate}
  \textit{We need to find the probability of the employee arriving early given that he leaves late: }

  \begin{align*}
    \Pr(A_l' |  L_e') &= \cfrac{\Pr(A_l' \cap L_e')}{\Pr(L_e')}\\\\
    & = \cfrac{\Pr(A_l \cup L_e)'}{\Pr(L_e')}\\\\
    & = \cfrac{(\Pr(A_l) + \Pr(L_e) - \Pr(A_l \cup L_e))'}{\Pr(L_e')}\\\\
    & = \cfrac{1-(\Pr(A_l) + \Pr(L_e) - \Pr(A_l \cup L_e))}{1 - \Pr(L_e)}\\\\
    & = \cfrac{1 - (0.15 + 0.25 - 0.08)}{0.75}\\\\
    & = 0.9066666667\\
  \end{align*}
  \[
     \boxed{\therefore\ \Pr(A_l' |  L_e')\ \ = \ \ 0.9067}\\
  \]

  \textit{Hence, the probability of the employee arriving early given that he leaves late is 80\%.}\\\\

  \newpage
  \item\textnormal{Given S = \{ 1, 2, ..., n\}, and X is a subset of S where if coins lands a heads then that particular element is added to X, and otherwise not.}\\
  \begin{enumerate}
    \item \textnormal{For each coin toss there are two possible outcomes: either it is included in X or not.}\\
    \textnormal{Given that a fair coin is tossed and all tosses are independent, the probability of it being}\\
    \textnormal{in X is ($\cfrac{1}{2})$ and that of not being in X is ($\cfrac{1}{2})$.}\\\\
    \textnormal{As there are total \it n \normalfont element/coins in the set S, and each element has 2 possible outcomes,}\\
    \textnormal{Total number of possible outcomes will be: $2^n$.}\\\\
    \textnormal{Since each outcome is equally likely to be include in set X or not, i.e ($\cfrac{1}{2})$,}
    \textnormal{and there are total of $2^n$ outcomes, and each of them having equal probability of occuring, i.e ($\cfrac{1}{2})^n$.}\\\\
    \textnormal{$\therefore$ Set X  is equally likely to be any one of the $2^n$ possible subsets.}\\
    \item \textnormal{X and Y are two sets choosen independently and uniformly at random from $2^n$ subsets of set S.}\\
    \textit{Note: here X and Y are not representing the outcomes of individual coin flips, they represent two subsets randomly chosen from set S.}\\
    \begin{enumerate}
      \item $\Pr(X \subseteq Y)$:\\
      \textnormal{For each element in the set S, there are two possibilities, it is included in X or it is not.}
      \textnormal{Similarly, there are two possibilities for each element regarding Y, it is included in Y or it is not.}\\
      \textnormal{Probability that a specific element is in X is ($\cfrac{1}{2})$, and that it is in Y is also ($\cfrac{1}{2})$.}\\
      \textnormal{Now for X to be subset of Y, every element in X must also be ther in Y, Hence Probaility that}
      \textnormal{a element is in both is: $(\cfrac{1}{2})(\cfrac{1}{2}) = \cfrac{1}{4}$.}\\\\
      \[
        \boxed{\therefore\  \Pr(X \subseteq Y) = (\cfrac{1}{4})^n}\\
      \]\textit{(as there as n elements)}\\
      \item $\Pr(X \cup Y = \{1, 2,..., n\})$:\\
      \textnormal{Probability that a specific element is in either of X or Y, is the sum of probability of a element being in X and probability of a element being in Y.}\\
      Applying the complement rule

      \begin{align*}
        \Pr(X \cup Y) = 1 - \Pr(X \cup Y)^{'}\\
        \Pr(X \cup Y)^{'} = \Pr(X^{'} \cap Y^{'})
      \end{align*}
      \begin{align*}
        \Pr(X \cup Y = \{1, 2,..., n\}) & = 1 - \Pr(X^{'} \cap\ Y^{'})\\
        &= 1 - \Pr(X^{'}) \Pr(Y^{'})\\
        &= 1 - (\cfrac{1}{2})^{n}(\cfrac{1}{2})^{n}\\\\
        &= \left(\cfrac{3}{4}\right)^{n}
      \end{align*}
      \[
        \boxed{\therefore \Pr(X \cup Y = \{1, 2,..., n\}) = \left(\cfrac{3}{4}\right)^{n}}\\
      \]
      \textit{Hence, it is certain that union of X and Y is indeed set S.}
    \end{enumerate}
  \end{enumerate}
\newpage
\item\textnormal{We know that there are several different min-cut sets in the graph.}
    
    \textnormal{Let us consider $C_1, ...,C_k$ be the distinct minimum cuts of the graph.}\\
    \textnormal{Let $\mathcal{E}_i$ be the event that $C_i$ is output using the analysis of the randomized min-cut algorithm.}\\
    \textnormal{Since the event $\mathcal{E}_i$ is disjoint, it makes all these randomized events disjoint as follows:}\\
    \begin{center}
        $\sum_{i,j} \ \Pr[\mathcal{E}_i] \leq 1.$
    \end{center}
    \textnormal{By the analysis of the randomized min-cut algorithm, is showed that:}\\
    \begin{center}
        $\Pr[\mathcal{E}_i] = \frac{n(n-1)}{2}$
    \end{center}
    \textnormal{for every $i$, which then implies that}\\
    \begin{center}
        $k \leq \frac{n(n-1)}{2}.$
    \end{center}
    \textnormal{This holds true as the n-cycle has exactly $\binom{n}{2}$ minimum cuts.}\\
    \textnormal{Hence, from the above explanation, it is concluded that there can be at most $\frac{n(n-1)}{2}$ distinct min cut-sets in a graph.}\\
\newpage
% \item \textbf{Inclusion-Exclusion Principle}
%   \begin{align*}
%     \Pr\left(\bigcup_{i=1}^{n} E_i\right) = \sum_{i=1}^{n} \Pr(E_i) - \sum_{i<j}^{} \Pr(E_i E_j) + \sum_{i<j<k}^{} \Pr(E_i E_j E_k) - ... + (-1)^{n+1}\Pr(E_1..... E_n)
%   \end{align*}
%   \textnormal{This formula consists of alternating sums of the sizes of intersections of the sets.}
\item
\item[(a)]\begin{equation*}
  P\left(\bigcup_{i=1}^{n} E_i\right) = \sum_{i=1}^{n} P(E_i) - \sum_{\substack{i,j=1 \\ i < j}}^{n} P(E_i \cap E_j) + \ldots + (-1)^{l+1} \sum_{\substack{i_1,\ldots,i_l=1 \\ i_1 < \cdots < i_l}}^{n} P\left(\bigcap_{r=1}^{l} E_{i_r}\right) + \ldots
  \end{equation*}
  
  It is given that \( E_1, \ldots, E_n \) be the \( n \) events.
  
  Let \( l \) is odd, then for events \( E_1, \ldots, E_n \), the relation is as follows:
  \begin{equation*}
  P\left(\bigcup_{i=1}^{l} E_i\right) \leq \sum_{i=1}^{l} P(E_i) - \sum_{\substack{i,j=1 \\ i < j}}^{l} P(E_i \cap E_j) + \ldots + (-1)^{l+1} \sum_{\substack{i_1,\ldots,i_l=1 \\ i_1 < \cdots < i_l}}^{l} P\left(\bigcap_{r=1}^{l} E_{i_r}\right) \ldots \quad (1)
  \end{equation*}
  
  Now, let \( l \) is even, then for events \( E_1, \ldots, E_n \), the relation is as follows:
  \begin{equation*}
  P\left(\bigcup_{i=1}^{l} E_i\right) \geq \sum_{i=1}^{l} P(E_i) - \sum_{\substack{i,j=1 \\ i < j}}^{l} P(E_i \cap E_j) + \ldots + (-1)^{l+1} \sum_{\substack{i_1,\ldots,i_l=1 \\ i_1 < \cdots < i_l}}^{l} P\left(\bigcap_{r=1}^{l} E_{i_r}\right) \ldots \quad (2)
  \end{equation*}
  
  From equation (1) and (2), the relation is as follows:
  \begin{equation*}
  P\left(\bigcup_{i=1}^{n} E_i\right) = \sum_{i=1}^{n} P(E_i) - \sum_{\substack{i,j=1 \\ i < j}}^{n} P(E_i \cap E_j) + \ldots + (-1)^{l+1} \sum_{\substack{i_1,\ldots,i_l=1 \\ i_1 < \cdots < i_l}}^{n} P\left(\bigcap_{r=1}^{l} E_{i_r}\right) + \ldots
  \end{equation*}
  
  Hence, inclusion-exclusion principle is proved. 
  \item[(b)]
  \begin{equation*}
  P\left(\bigcup_{i=1}^{l} E_i\right) \leq \sum_{i=1}^{l} P(E_i) - \sum_{1 \leq i < j \leq l} P(E_i \cap E_j) + \ldots + (-1)^{l+1} \sum_{1 \leq i_1 < i_2 < \dots < i_l \leq l} P\left(E_{i_1}\cap...\cap E_{i_k}\right)
  \end{equation*}
  It is given that \( l \) is odd.\\
  According to the Bonferroni inequality which states that for odd \( k \) in \( 1,2,\ldots,n \),
  \begin{center}
    $P\left(\bigcup_{i=1}^{l} E_i\right) \leq \sum_{j=1}^{k} (-1)^{j+1} S_j,$  
  \end{center}
  , and for even \( k \) in \( 1,2,\ldots,n \),
  \begin{center}
      $P\left(\bigcup_{i=1}^{l} E_i\right) \geq \sum_{j=1}^{k} (-1)^{j+1} S_j$,
  \end{center}

  where,
\begin{equation*}
S_1 = \sum_{i=1}^{l} \Pr(E_i), \quad S_2 = \sum_{1 \leq i < j \leq l} \Pr(E_i \cap E_j), \quad \text{and} \quad S_k = \sum_{1 \leq i_1 < i_2 < \dots < i_k \leq l} \Pr(E_{i_1}\cap...\cap E_{i_k}) \text{ for } k = 3,4,\ldots,n.
\end{equation*}

Expand the right-hand side of the inequality
\begin{equation*}
P\left(\bigcup_{i=1}^{l} E_i\right) \leq \sum_{j=1}^{k} (-1)^{j+1} S_j,
\end{equation*}
as follows:
\begin{equation*}
(-1)^1 S_1 + (-1)^2 S_2 + (-1)^3 S_3 + \ldots + (-1)^k S_k = S_1 - S_2 + S_3 - \ldots + (-1)^k S_k.
\end{equation*}

Solve the above expression further as follows:
\begin{equation*}
P\left(\bigcup_{i=1}^{l} E_i\right) \leq S_1 - S_2 + S_3 - \ldots + (-1)^k S_k.
\end{equation*}

Substitute \( S_j \) in the above expression. It is found that
\begin{equation*}
P\left(\bigcup_{i=1}^{l} E_i\right) \leq \sum_{i=1}^{l} P(E_i) - \sum_{1 \leq i < j \leq l} P(E_i \cap E_j) + \ldots + (-1)^{l+1} \sum_{1 \leq i_1 < i_2 < \dots < i_l \leq l} P\left(E_{i_1}\cap...\cap E_{i_l}\right).
\end{equation*}

Hence, the inequality for \( l \) odd is proved.
\item[(c)] Recall the Bonferroni inequality which states that for odd \( k \) in \( 1,2,\ldots,n \),
\begin{equation*}
P\left(\bigcup_{i=1}^{l} E_i\right) \leq \sum_{j=1}^{k} (-1)^{j+1} S_j,
\end{equation*}
and for even \( k \) in \( 1,2,\ldots,n \),
\begin{equation*}
P\left(\bigcup_{i=1}^{l} E_i\right) \geq \sum_{j=1}^{k} (-1)^{j+1} S_j,
\end{equation*}
where,
\begin{equation*}
S_1 = \sum_{i=1}^{l} P(E_i), \quad S_2 = \sum_{1 \leq i < j \leq l} P(E_i \cap E_j), \quad S_k = \sum_{1 \leq i_1 < i_2 < \dots < i_k \leq l} P\left(E_{i_1}\cap...\cap E_{i_k}\right) \text{ for } k = 3,4,\ldots,n.
\end{equation*}

Expand the right-hand side of the inequality
\begin{equation*}
P\left(\bigcup_{i=1}^{l} E_i\right) \geq \sum_{j=1}^{k} (-1)^{j+1} S_j,
\end{equation*}
as follows:
\begin{equation*}
(-1)^1 S_1 + (-1)^2 S_2 + (-1)^3 S_3 + \ldots + (-1)^k S_k = S_1 - S_2 + S_3 - \ldots + (-1)^k S_k.
\end{equation*}

Solve the above expression further as follows:
\begin{equation*}
P\left(\bigcup_{i=1}^{l} E_i\right) \geq S_1 - S_2 + S_3 - \ldots + (-1)^{k} S_k.
\end{equation*}

Substitute \( S_j \) in the above expression. It is found that
\begin{equation*}
P\left(\bigcup_{i=1}^{l} E_i\right) \geq \sum_{i=1}^{l} P(E_i) - \sum_{1 \leq i < j \leq l} P(E_i \cap E_j) + \ldots + (-1)^{k} \sum_{1 \leq i_1 < i_2 < \dots < i_k \leq l} P\left(E_{i_1}\cap...\cap E_{i_l}\right).
\end{equation*}

Hence, the Bonferroni inequality is proved.
\end{enumerate}
\end{document}
