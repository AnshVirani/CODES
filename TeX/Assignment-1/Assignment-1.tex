\documentclass{article}
\newcounter{rownumbers}
\newcommand\rownumber{\stepcounter{rownumbers}\arabic{rownumbers}}
%\usepackage[a4paper, total={6in, 8in}]{geometry}
\usepackage{graphicx}
\usepackage{bm}
\usepackage{amsmath,amsfonts,mathtools}

\graphicspath{ {C:\Users\harsh\OneDrive\Documents} }
\usepackage{geometry}
 \geometry{
 a4paper,
 total={210mm,297mm},
 left=20mm,
 right=20mm,
 top=-2mm,
 bottom=2mm,
 }
 
%\usepackage[margin=0.5in]{geometry}

\usepackage{amsmath,amssymb}
\usepackage{ifpdf}
%\usepackage{cite}
\usepackage{algorithmic}
\usepackage{array}
\usepackage{mdwmath}
\usepackage{pdfpages}
\usepackage{mdwtab}
\usepackage{eqparbox}
\usepackage{parskip}
%\onecolumn
%\input{psfig}
\usepackage{color}
\usepackage{graphicx}
\setlength{\textheight}{23.5cm} \setlength{\topmargin}{-1.05cm}
\setlength{\textwidth}{6.5in} \setlength{\oddsidemargin}{-0.5cm}
\renewcommand{\baselinestretch}{1}
\pagenumbering{arabic}
\linespread{1.15}
\begin{document}
\textbf{
\begin{center}
{
\large{School of Engineering and Applied Science (SEAS), Ahmedabad University}\vspace{5mm}
}
\end{center}
%
\begin{center}
\large{Probability and Stochastic Processes (MAT277)\\ \vspace{4mm}
Homework Assignment-1\\\vspace{2mm}
Enrollment No: AU2140096 \hspace{4cm} Name: Ansh Virani }
\end{center}}
\vspace{2mm}


\vspace{2mm}

\begin{enumerate}
\item \textbf{While tossing a biased die, calculate the probability that
face 3 has turned up, Given Alex tells either face 3 or face 6 has turned up.}

\begin{enumerate}
  \item\textit{We are given that,}
        \begin{center}
        \begin{tabular}{ | m{5em} | m{1cm}| m{1cm} | m{1cm} | m{1cm} | m{1cm} | m{1cm} | }
        \hline
        Face & 1 & 2 & 3 & 4 & 5 & 6\\
        \hline
        Probability & 0.2 & 0.22 & 0.11 & 0.25 & 0.15 & 0.07\\
        \hline
        \end{tabular}
        \end{center}

        \textit{Let A be the event that face 3 has turned up and B be the event that face 6 has turned up.}\\\\
        $\therefore $ $P(A) = 0.11\ \ \ \&\ \ $ $P(B) = 0.07$\\\\
        \textit{We know that these two are mutually exclusive events, hence:}\\\\
        $\begin{aligned}
        \therefore P(A \cup B) & = P(A) + P(B)\\
        & = 0.11 + 0.32 \\
        & = 0.43
        \end{aligned}$\\\\
        \textit{Clearly, here we have to find the conditional probability, $P(A \ |\ A \cup B)$}

        \begin{align*}
        P(A \ |\ A \cup B) & = \frac{P(A \cap (A \cup B))}{P(A \cup B)}\\\\
        & = \frac{P(A)}{P(A \cup B)}\\\\
        & = \frac{0.11}{0.43}\\\\
        & = 0.2558139535.
        \end{align*}

        \[
          \boxed{\therefore\ P(A \ |\ A \cup B)\ \ \approx \ \ 0.2559}
        \]
        
        \textit{Hence, the probability that face 3 has turned up, given either face 3 or face 6 has turned up is approx 25.59\%}
\end{enumerate}
\newpage
\item \textbf{There exists two events E1 \& E2 such that $P(E1 \ |\ E2) = 0.45$, $P(E2 \ |\ E1) = 0.5$ and $P(E1 \cup E2) = 0.4$}
  \begin{enumerate}
    \item\textit{Calculate $P(E1 \cap E2$):}\\\\
    \textit{According to Conditional Probability theorem we can say,}\\\\
    $\begin{aligned}
      P(E1 \ |\ E2)P(E2) & = P(E2 \ |\ E1)P(E1)\\
      0.45P(E2) & = 0.5P(E1) \\
    \end{aligned}$\\\\
    \textit{Applying Basic Probaility theorem:}\\\\
    $\begin{aligned}
      P(E1 \cup E2) & = P(E1) + P(E2) + P(E1 \cap E2)\\
      0.4 & = 0.9P(E2) + P(E2) + P(E1 \cap E2)\\\\
    \end{aligned}$

    $\therefore P(E1 \cap E2) = 1.9P(E2) - 0.4$ \hspace{8cm}......(1)\\\\
    \textit{Now, Substituting these equation in conditional probability equation:}\\\\
    $\begin{aligned}
      P(E1 \ |\ E2) & = \frac{P(E1 \cap E2)}{P(E2)}\\\\
      0.45 & = \frac{1.9P(E2) - 0.4}{P(E2)}\\
      0.4 & = 1.45P(E2)\\\\
      \therefore P(E2) = 0.275
    \end{aligned}$\\\\
    \textit{Substituting value of P(E2) in equation 1, we get}\\\\
    $\therefore P(E1) = 0.2475$\\\\
    \item \textit{ Comment on the dependency relation between event E1 and E2:}
    
    \textnormal{When event E2 occurs, there's a 45\% chance that event E1 will occur. On the other hand if event E1 occurs, there's a 50\% chance that event E2 will occur.}
    \textnormal{This shows that the events are unsymmetrically related.}
    
    \textnormal{Also upon calculating it was found that, there's some shared occurance happening beteen two of the given events,}
    \textnormal{as probability of them happening together was founf to be 12.25\%}
  \end{enumerate}

\newpage
\item\textbf{Given Probabilies are:}\\
  \begin{enumerate}
    \item\textbf{Let's denote the probability of a Red ball as:  $P(R) = 0.45$}
    \item\textbf{Let's denote the probability of a Striped ball as:  $P(S) = 0.3$}
    \item\textbf{Let's denote the probability of a Red ball with stripes as:  $P(RS) = P(R \cap S) = 0.2$}\\
  \end{enumerate}
  \textit{To find the probability that ball is striped given the ball picked is a Red one.}\\
  \begin{align*}
    P(S \ |\ R) & = \frac{P(R \cap S)}{P(R)}\\\\
    & = \frac{0.2}{0.45}\\\\
    & = 0.4444444...
  \end{align*}\\
  \[
     \boxed{\therefore\ P(A \ |\ A \cup B)\ \ \approx \ \ 0.4445}\\
  \]

  \textit{Hence, the probability that the ball is striped one given the ball in red ball is approx 44.45\%}\\\\

\newpage  
\item \textbf{Let A denote the event where number 8 is obtained while tossing a 8-sided unbiased dice}\\
$\therefore P(A) = \ \cfrac{1}{8}$ = \ p\\\\
$\therefore P(A') = 1 - p $  

\textnormal{X denotes the number of tosses required to get number 8 as an outcome.}\\
\begin{enumerate}
    \item \textnormal{The probability that X = 6:}\\
    \textit{we use the equation:}
    \begin{align*}
      P(X) & = P(A')^{(X-1)}.P(A)\\\\
      P(X_6)\  & =\  (1-p)^{5}.p\\
      & =\ (\cfrac{7}{8})^{5}.\cfrac{1}{8}\\
      & =\ (0.875)^{5}*(0.125)\\
      & =\ 0.0641136169
    \end{align*}\\
    \[
     \boxed{\therefore\ P(X_6)\ \ \approx \ \ 0.06412}\\
    \]\\
    \item \textnormal{Conditional Probaility that $X \leq 6$ given $X < 9$:}\\\\
    \begin{align*}
      P(X\leq 6 \ | \ X < 9) & = \cfrac{P(X\leq 6 \cap X < 9)}{P(X < 9)}\\\\
      & = \cfrac{P(X < 9)}{P(X < 9)}
    \end{align*}\\
    \[
     \boxed{\therefore\ P(X\leq 6 \ | \ X < 9)\ \ = \ \ 1}\\
    \]

    \textit{Hence, as the equation suggests both the cases are same, resulting 100\% probability.}\\\\

  \end{enumerate}

\newpage
\item\textbf{Given Probabilies are:}\\
  \begin{enumerate}
    \item\textbf{probability that an employee arrives late:  $P(A_l) = 0.15$}
    \item\textbf{probability that an employee leaves early:  $P(L_e) = 0.25$}
    \item\textbf{probability that an employee arrives late and leaves early:  $P(A_l \cap L_e) = 0.8$}\\
  \end{enumerate}
  \textit{We need to find the probability of the employee arriving early given that he leaves late: }

  \begin{align*}
    P(A_l' |  L_e') &= \cfrac{P(A_l' \cap L_e')}{P(L_e')}\\\\
    & = \cfrac{P(A_l \cup L_e)'}{P(L_e')}\\\\
    & = \cfrac{(P(A_l) + P(L_e) - P(A_l \cup L_e))'}{P(L_e')}\\\\
    & = \cfrac{1-(P(A_l) + P(L_e) - P(A_l \cup L_e))}{1 - P(L_e)}\\\\
    & = \cfrac{1 - (0.15 + 0.25 - 0.8)}{0.75}\\\\
    & = 0.8\\
  \end{align*}
  \[
     \boxed{\therefore\ P(A_l' |  L_e')\ \ = \ \ 0.8}\\
  \]

  \textit{Hence, the probability of the employee arriving early given that he leaves late is 80\%.}\\\\


\end{enumerate}
\end{document}
