\documentclass{article}
\newcounter{rownumbers}
\newcommand\rownumber{\stepcounter{rownumbers}\arabic{rownumbers}}
%\usepackage[a4paper, total={6in, 8in}]{geometry}
\usepackage{graphicx}
\usepackage{bm}
\usepackage{amsmath,amsfonts,mathtools}

\graphicspath{ {C:\Users\harsh\OneDrive\Documents} }
\usepackage{geometry}
 \geometry{
 a4paper,
 total={210mm,297mm},
 left=20mm,
 right=20mm,
 top=-2mm,
 bottom=2mm,
 }
 
%\usepackage[margin=0.5in]{geometry}

\usepackage{amsmath,amssymb}
\usepackage{ifpdf}
%\usepackage{cite}
\usepackage{algorithmic}
\usepackage{array}
\usepackage{mdwmath}
\usepackage{pdfpages}
\usepackage{mdwtab}
\usepackage{eqparbox}
\usepackage{parskip}
%\onecolumn
%\input{psfig}
\usepackage{color}
\usepackage{graphicx}
\setlength{\textheight}{23.5cm} \setlength{\topmargin}{-1.05cm}
\setlength{\textwidth}{6.5in} \setlength{\oddsidemargin}{-0.5cm}
\renewcommand{\baselinestretch}{1}
\pagenumbering{arabic}
\linespread{1.15}
\begin{document}
\textbf{
\begin{center}
{
\large{School of Engineering and Applied Science (SEAS), Ahmedabad University}\vspace{5mm}
}
\end{center}
%
\begin{center}
\large{Probability and Stochastic Processes (MAT277)\\ \vspace{4mm}
Homework Assignment-3\\\vspace{2mm}
Enrollment No: AU2140096 \hspace{4cm} Name: Ansh Virani }
\end{center}}
\vspace{2mm}


\vspace{10mm}

\begin{enumerate}
\item 
    Consider that Alice and Bob have no first girl child then they have \(k\) children as mentioned in the question. So, the probability for no girl child is \( \left(\cfrac{1}{2^k}\right)\).\\\\
    When they have one girl the probability is the complement of no girl child,\\
    \[
    \left(1 - \cfrac{1}{2^k}\right)
    \]
    The first \( i^{th} \) offspring are usually boys, that is to have at least \( i \geq 1 \) the probability is \( \cfrac{1}{2^i} \).\\
    Therefore, the predicted number of male children has increased as
    \[
    \sum_{i=1}^k \left(\cfrac{1}{2^i}\right) = \left(1 - \cfrac{1}{2^k}\right)
    \]
    \[
        \boxed{\therefore\ \textnormal{The expected number of female or male children is same that is } \left(1 - \cfrac{1}{2^k}\right).}
    \]
\newpage
\item 
\begin{enumerate}
    \item 
    \textbf{Maximizing the expected amount of money:}\\
    Given that the commodity sells for \$2 per ounce now, we have \$1000 to spend. \\
    Therefore, we can buy 
    \[ 
    \cfrac{1,000}{2} = 500 \text{ ounces}
    \]
    After one week, the price of the commodity can either increase to \$4 per ounce or decrease to \$1 per ounce. Both possibilities are equally likely.\\
    \begin{enumerate}
        \item If the price increases to \$4 per ounce:
        \begin{itemize}
            \item we have 500 ounces.
            \item Selling them at \$4 per ounce would give us \( 500 \times 4 = \$2,000. \)
        \end{itemize}
        \item If the price decreases to \$1 per ounce:
        \begin{itemize}
            \item we still have 500 ounces.
            \item Selling them at \$1 per ounce would give us \( 500 \times 1 = \$500. \)
        \end{itemize}
    \end{enumerate}
    Since both the scenarios are equally likely to occur, we calculate the expected value by taking the average of the possible outcomes:
    \[
        E [\text{Money at the end of the week}] = \cfrac{2,000 + 500}{2} = \$1,250
    \]
    \[
        \boxed{\therefore\ \textnormal{Our expected amount of the commodity at the end of the week will be \(\approx\) 1,250 ounces.}}
    \]
    \item \textbf{Maximizing the expected amount of the commodity at the end of the wek:}\\
    We should wait until the end of the week to buy because the price will either be \$1 or \$4 per ounce with equal probability.\\\\
    Therefore, calculating the expected price at the end of the week by taking average we get, \$2.50 per ounce.\\    
    With the given amount, we can buy
    \[
    \cfrac{1,000}{2.50} = 400 \text{ ounces}
    \]
    \[
        \boxed{\therefore\ \textnormal{Our expected amount of the commodity at the end of the week will be 400 ounces.}}
    \]
    Hence, if our objective is to maximize the expected amount of money, we should buy the commodity now and hold onto it.\\\\
    That strategy ensures us obtain the most commodity for your money, taking into account the uncertainty that can take place within the price fluctuations.\\
    \end{enumerate}

\newpage
\item 
    Let, the variable \( X \) denotes the player's winnings. The probability that the player wins their \( n^{th} \) game is given by \( \left(\cfrac{1}{2}\right)^n \).\\
    Calculating the Expectation:
        \begin{align*}
        E [X] &= \sum_{i=1}^{\infty} X P(X = X) \\\\
            &= \sum_{i=1}^{\infty} \left[2^n\cdot \left(\frac{1}{2}\right)^n\right] \\\\
            &= \sum_{i=1}^{\infty} (1) \\\\
            &= \infty
        \end{align*}
        \begin{enumerate}
            \item 
            No, I would not be pay \$1 million to play this game once.\\
            Considering the best case scenario: \$1 million to play once and you win in the first toss, you will take home a big amount \$2 million. But on the other hand, if you lose the bet, you will lose the invested amount.(\textit{Not a small amount})\\\\
            \textit{The fact \( E[X] = \infty \) is the expected value in the long run, not for one game.}\\
            \item 
            Yes, the expectation when considering a long run is indeed "something large," so we can expect the profit. Therefore, one can pay \$1 million for each game.
        \end{enumerate}

\newpage
\item 
    From the properties of expectation,    
    \[
        E[a] = a, \quad \text{here } a \text{ is a constant.}
    \]
    \[
        E[a - X] = a - E[X], \quad \text{here } X \text{ is a random variable.}
    \]
    Consider any constant function \( y \) as function of random variable \( X \) as below.
    \[
        y = (X - E[X])^k \quad \ldots (1)
    \]  
    Expand R.H.S of eq.(1) using Binomial's expansion,
    \begin{align*}
        y &= (X - E[X])^k \\
          &= \binom{k}{0} X^k (-E[X])^0 + \binom{k}{1} X^{k-1}\cdot(-E[X])^1 + \ldots + \binom{k}{k-1} X^1\cdot(-E[X])^{k-1} \\
          &= X^k + \binom{k}{1} X^{k-1}\cdot(-E[X]) + \ldots + \binom{k}{k-1} X(-E[X])^{k-1}
    \end{align*}
    From the properties of expectation, 
    \begin{equation*}
        E[a] = a, \quad \text{here } a \text{ is aconstant.}
    \end{equation*}
    \begin{equation*}
        E[a - X] = a - E[X], \quad \text{here } X \text{ is a random variable.}
    \end{equation*} 
    Consider any constant function \( y \) as function of random variable \( X \) as below.
    \begin{equation}
        y = (X - E[X])^k \ldots (1)
    \end{equation}
        Expand Right hand side of equation (1) using Binomial's expansion,
    \begin{equation*}
        y = (X - E[X])^k = \binom{k}{0}X^k(-E[X])^0 + \binom{k}{1}X^{k-1}(-E[X])^1 + \ldots + \binom{k}{k}X^0(-E[X])^k
    \end{equation*}
    \begin{equation*}
        = X^k + \binom{k}{1}X^{k-1}(-E[X]) + \ldots + (-E[X])^k
    \end{equation*}
    From above expansion,
    \begin{equation*}
        y - X^k = (-E[X])^k - \binom{k}{1}X^{k-1}(-E[X]) + \ldots + \binom{k}{k-1}X(-E[X])^{k-1} \ldots (2)
    \end{equation*} 
    Apply expectation on both sides of equation (2). 
    \begin{equation*}
        E[y - X^k] = E[(-E[X])^k - \binom{k}{1}X^{k-1}(-E[X]) + \ldots + \binom{k}{k-1}X(-E[X])^{k-1}] \ldots (3)
    \end{equation*}  
    On applying properties of expectation from the equation (3).  
    \begin{equation*}
        E[X^k] = (E[X])^k - \binom{k}{1}E[X^{k-1}(-E[X])] + \ldots + \binom{k}{k-1}E[X(-E[X])^{k-1}] \ldots (4)
    \end{equation*}
    From above expression (4) it is interpreted that, 
    \begin{equation*}
        (E[X])^k > (E[X])^k
    \end{equation*}
    Hence, the required objective is proved.
\end{enumerate}
\end{document}