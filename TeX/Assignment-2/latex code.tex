\documentclass{article}
\newcounter{rownumbers}
\newcommand\rownumber{\stepcounter{rownumbers}\arabic{rownumbers}}
%\usepackage[a4paper, total={6in, 8in}]{geometry}
\usepackage{graphicx}
\usepackage{bm}
\usepackage{amsmath,amsfonts,mathtools}

\graphicspath{ {C:\Users\harsh\OneDrive\Documents} }
\usepackage{geometry}
 \geometry{
 a4paper,
 total={210mm,297mm},
 left=20mm,
 right=20mm,
 top=-2mm,
 bottom=2mm,
 }
 
%\usepackage[margin=0.5in]{geometry}

\usepackage{amsmath,amssymb}
\usepackage{ifpdf}
%\usepackage{cite}
\usepackage{algorithmic}
\usepackage{array}
\usepackage{mdwmath}
\usepackage{pdfpages}
\usepackage{mdwtab}
\usepackage{eqparbox}
\usepackage{parskip}
%\onecolumn
%\input{psfig}
\usepackage{color}
\usepackage{graphicx}
\setlength{\textheight}{23.5cm} \setlength{\topmargin}{-1.05cm}
\setlength{\textwidth}{6.5in} \setlength{\oddsidemargin}{-0.5cm}
\renewcommand{\baselinestretch}{1}
\pagenumbering{arabic}
\linespread{1.15}
\begin{document}
\textbf{
\begin{center}
{
\large{School of Engineering and Applied Science (SEAS), Ahmedabad University}\vspace{4mm}
}
\end{center}
%
\begin{center}
\large{Probability and Stochastic Processes (MAT277)\\ \vspace{4mm}
Homework Assignment-2\\\vspace{2mm}
Enrollment No: AU2140096 \hspace{4cm} Name: Ansh Virani }
\end{center}}
\vspace{2mm}

\vspace{10mm}

\begin{enumerate}
\item
  \textbf{Let event A be the probability that Omega is significant: $\Pr(A) = 0.75$}\\
  \textbf{Let there be an event that the Algorithm Delta produces a positive result, given that Omega is significant: $\Pr(B|A) = 1-0.15 = 0.85$}\\
  \textbf{Let there be an event that the Algorithm Delta produces a positive result, given that Omega is insignificant: $\Pr(B|A') = 0.15$}\\
  \textbf{The probability that Omega is insignificant: $\Pr(A') = 0.25$}\\\\
  \textit{The Probability that the message is significant given Algorithm Delta is positive is,}
  \begin{align*}
    Pr(A|B) &= \frac{Pr(B|A).Pr(A)}{Pr(B|A).Pr(A) + Pr(B|A').Pr(A')}\\\\
    &= \frac{(0.85). (0.75)}{(0.85). (0.75) + (0.15). (0.25)}\\\\
    &= \frac{0.6375}{0.675}\\\\
    &= 0.9444444
  \end{align*}
  \[
     \boxed{\therefore\ \Pr(A|B )\ \ \approx \ \ 0.945}\\
  \]

  \textit{Hence, as the probability that message contains critical information is less than 95\%, we should opt for further analysis.}\\\\

\newpage
\item \textbf{An undirected graph $G$ with $n$ nodes is given.}\\
  \textnormal{The min-cut algorithm works by iteratively contracting edges until only two nodes remain, representing the two disjoint sets of the minimum cut. The contraction process merges nodes, and the algorithm repeats until only two super-nodes are left.}\\
  \begin{enumerate}
    \item \textnormal{The total number of edges in the graph is given by the binomial coefficient \( \binom{n}{2} \), which represents all possible ways to choose 2 nodes out of n. This is equal to \(\cfrac{n(n-1)}{2}\).}\\\\
    \textnormal{During the contraction process, each contraction operation results in a unique cut. Since there are \(\cfrac{n(n-1)}{2}\) distinct edges in the graph, there can be at most \(\cfrac{n(n-1)}{2}\) distinct cuts formed during the contraction process.}\\
    \(\therefore\)  we can argue that there can be at most \(\cfrac{n(n-1)}{2}\) distinct min-cut sets in a graph $G$ with $n$ nodes using the analysis of the min-cut algorithm.\\
    \item We consider the analysis of the min-cut Algorithm, which indicates that for a graph to have a min-cut of size $k$, each vertex in the graph must have $k$ as their degree, $i.e$ removing $k$ edges should disconnect the entire graph.\\\\
    Since it's given that the graph is undirected, the number of vertices will be twice the number of edges.\\
    If each of the $n$ vertices has a degree of atleast $k$, the total degree count is atleast ($n\cdot k$)\\\\
    However, each edge has been counted twice as it connects two vertices, the actual number chqanges to: ($\cfrac{n\cdot k}{2}$).\\\\
    Hence, if Graph $G$ has min-cut cardinality $k$, then it has atleast $\left(\cfrac{n\cdot k}{2}\right)$ edges.
  \end{enumerate}

\newpage
\item \textbf{Let A denote the event where number 6 is obtained while tossing a 6-sided unbiased dice.}\\
  \begin{enumerate}
    \item \textnormal{The probability of rolling a 6 on a single trial is: \(\Pr(A) = \cfrac{1}{6}\)}\\\\
    \textnormal{The probability of not rolling a 6 on a single trial is: \(\Pr(A') = \cfrac{5}{6}\)}\\\\
    Now, let us consider the event that a six occurs on the \(k^{th}\) roll. This implies that there were \(k-1\) consecutive non-six outcomes followed by a six. The probability of this sequence is given by:
    \begin{align*}
      \Pr(X= k) & = \Pr(A')^{(X-1)}.\Pr(A)\\\\
      \Pr(X= k)& =\ \left(\cfrac{5}{6}\right)^{k-1}\cdot \left(\cfrac{1}{6}\right)\\
    \end{align*}
    \item To Check the legitimacy of the PMF, we need to satisfy two conditions:
      \begin{enumerate}
        \item Each probability of P(X = k) is non-negative for all possible values of \(k\).
        \begin{align*}
          \Pr(X=k) \geq 0\\\\
          \left(\cfrac{5}{6}\right)^{k-1}\cdot \left(\cfrac{1}{6}\right) \geq 0
        \end{align*}
        This is ture because both the terms, \(\left(\cfrac{5}{6}\right)^{k-1}\) and \(\left(\cfrac{1}{6}\right)\) are non-negative.\\
        \item The sum of all probabilities over all possible values of \(k\) is equal to 1.
        \begin{align*}
          \sum_{k=1}^{n} P(X=k)\ =\ \sum_{k=1}^{n} \left(\cfrac{5}{6}\right)^{k-1}\cdot \left(\cfrac{1}{6}\right)
        \end{align*}
        It's a geometric series where the sum can be represented as:\\
        \begin{align*}
          \sum_{k=1}^{n} \left(\frac{5}{6}\right)^{k-1} \cdot \left(\frac{1}{6}\right) &= \left(\frac{1}{6}\right) \cdot \frac{1 - \left(\frac{5}{6}\right)^n}{1 - \frac{5}{6}} \\
          &= 1 - \left(\frac{5}{6}\right)^n
      \end{align*}
      This term approaches 1 as \(n\) tends to infinity, the sum of probabilities converges to 1.\\\\
      $\therefore$ This PMF is legitimate.
      \end{enumerate}
  \end{enumerate}

\newpage
\item \textbf{The PMF of random variable X is given by:}
\begin{align*}
  p(i) = \cfrac{c.\lambda^i}{i!},\ \ \ \ i = 0, 1, 2,...\\
\end{align*}
\textnormal{Given $\lambda$ is a positive value.}
  \begin{enumerate}
    \item P\{X=0\} can be obtained by simply Substituting \(i\) = 0 in the above equation.
    \begin{align*}
      P\{X=0\} &= p(0) = \cfrac{c.\lambda^0}{0!}\\\\
      &= c
    \end{align*}
    \[
     \boxed{\therefore P\{X=0\} = c}\\
    \]\\
    \item P\{X \(>\) 2\} can also be written in other form as: 1 - P\{X \(\leq\) 2\}
    \begin{align*}
      P\{X\leq 2\} &= p(0) + p(1) + p(2)\\\\
      &= c + \cfrac{c.\lambda^1}{1!} + \cfrac{c.\lambda^2}{2!}\\\\
      P\{X > 2\} &= 1 - \left(c + c.\lambda + \cfrac{c.\lambda^2}{2}\right)\\
    \end{align*}
    \[
     \boxed{\therefore P\{X > 2\} = 1 - \left(\cfrac{c.\lambda^2 + 2c.\lambda + 2c}{2}\right)}\\
    \]
  \end{enumerate}
\newpage
\item 
  The event that the contestant knows the Answer to Question-1: \(E_1\)\\
  The event that the contestant knows the Answer to Question-2: \(E_2\)\\
  The random variable representing winnings for answering Question-1 correctly: \(X_1\)\\
  The random variable representing winnings for answering Question-2 correctly: \(X_2\)\\
  The amount won for answering Question-1 correctly: \(V_1\)\\
  The amount won for answering Question-2 correctly: \(V_2\)\\
  The probability of knowing the answer to Question-1: \(P_1\)\\
  The probability of knowing the answer to Question-2: \(P_2\)\\\\
  \textbf{\textit{Case-i}} If they attempt Question-1 first:\\
  $\bullet$ Get Question 1 right and Question 2 right\\
  $\bullet$ Get Question 1 right and Question 2 wrong\\
  $\bullet$ Get Question 1 wrong\\
  So the expected winnings for attempting Question-1 first is given by:
  \begin{align*}
    P_1.P_2.\left(V_1 + V_2\right) +  P_1.\left(1 - P_2\right).\left(V_1\right) + \left(1 - P_1\right).0
  \end{align*}
  \textbf{\textit{Case-ii}} If they attempt Question-2 first:\\
  $\bullet$ Get Question 2 right and Question 1 right\\
  $\bullet$ Get Question 2 right and Question 1 wrong\\
  $\bullet$ Get Question 2 wrong\\
  So the expected winnings for attempting Question-2 first is given by:
  \begin{align*}
    P_2.P_1.\left(V_1 + V_2\right) +  P_2.\left(1 - P_1\right).\left(V_2\right) + \left(1 - P_2\right).0
  \end{align*}
  Hence, To maximize expected winnings, the contestant should attempt the question first that yields the higher expected value from these above two equations.

\newpage
\item \textbf{A PMF is given by}\\
  \begin{align*}
    P(X={i}) = log_{10} \left(\cfrac{i + 1}{i}\right), \ \ \ i = 1, 2, 3,..., 9
  \end{align*}
  \begin{enumerate}
    \item to prove,
    \begin{align*}
      \sum_{i=1}^{9} log_{10}\left(\frac{i + 1}{i}\right) = 1\\
    \end{align*}
    $\therefore$ Solving \textbf{L.H.S}:
    \begin{align*}
      \sum_{i=1}^{9} log_{10}\left(\frac{i + 1}{i}\right) &= log_{10}\left(\frac{2}{1}\right) + log_{10}\left(\frac{3}{2}\right) + ....\ log_{10}\left(\frac{10}{9}\right)\\
      &= log_{10}\left[\left(\cfrac{2}{1}\right)\cdot\left(\cfrac{3}{2}\right)\cdot ...\cdot\left(\cfrac{10}{9}\right)\right]\\
      &= log_{10}\left(10\right)\\
      &= 1\\
      &= \textbf{R.H.S}
    \end{align*}
    Hence, Proved.\\
    \item $P(X \leq j) $
    \begin{align*}
      \sum_{i=1}^{j} log_{10}\left(\frac{i + 1}{i}\right) &= log_{10}\left(\frac{2}{1}\right) + log_{10}\left(\frac{3}{2}\right) + ....\ log_{10}\left(\frac{j + 1}{j}\right)\\
      &= log_{10}\left[\left(\cfrac{2}{1}\right)\cdot\left(\cfrac{3}{2}\right)\cdot ...\cdot\left(\cfrac{j + 1}{j}\right)\right]\\
      &= log_{10}\left(j + 1\right)\\
    \end{align*}
    \[
     \boxed{\therefore P\{X \leq j\} = log_{10}\left(j + 1\right)}\\
    \]
  \end{enumerate}
\newpage
\item
  Let us denote \(G\) as the number of correct guesses,\\
  Let \(P(G=k)\) denote the probability of having exactly \(k\) correct guesses.\\
  This signifies that the guesses are continuously correct till \(k^{th}\) guess, which is incorrect.\\
  The probability of guessing correct card for the \(1^{st}\) time: \(\left(\cfrac{1}{n}\right)\),\\
  The probability of guessing correct card for the \(2^{nd}\) time: \(\left(\cfrac{1}{n-1}\right)\),\\
  .\\.\\.\\
  The probability of guessing correct card for the \(k^{th}\) time: \(\left(\cfrac{1}{n-k+1}\right)\),\\
  \begin{align*}
    P(G=k) &= \left(\cfrac{1}{n}\right)\cdot\left(\cfrac{1}{n-1}\right)\cdot\left(\cfrac{1}{n-2}\right)\cdot ... \cdot\left(\cfrac{1}{n-k+2}\right)\cdot\left(\cfrac{1}{n-k+1}\right)
  \end{align*}
  \textit{We can rewrite the product of denominator's as:}
  \begin{align*}
    n\cdot\left(n-1\right)\cdot\left(n-2\right)\cdot...\cdot\left(n-k+1\right) = \cfrac{n!}{(n-k)!}
  \end{align*}
  $\therefore$
  \begin{align*}
    P(G=k) &= \left(\cfrac{1}{n\cdot\left(n-1\right)\cdot\left(n-2\right)\cdot...\cdot\left(n-k+1\right)}\right)\\
    &= \left(\cfrac{1}{\cfrac{n!}{(n-k)!}}\right)\\\\
    &= \cfrac{(n-k)!}{n!}
  \end{align*}
  \[
     \boxed{\therefore P(G=k) = \cfrac{(n-k)!}{n!}}\\
  \]

\newpage
\item 
  From the distribution function of the random variable \(X\):\\\\
  For $1 \leq x < 3$:
  \begin{align*}
    P(X = 1) = F(1) - F(-\infty) = \cfrac{1}{4} - 0 = \cfrac{1}{4}
  \end{align*}
  For $3 \leq x < 4$:
  \begin{align*}
    P(X = 3) = F(3) - F(1) = \cfrac{5}{8} - \cfrac{1}{4} = \cfrac{3}{8}
  \end{align*}
  For $4 \leq x < 6$:
  \begin{align*}
    P(X = 4) = F(4) - F(3) = \cfrac{3}{4} - \cfrac{5}{8} = \cfrac{1}{8}
  \end{align*}
  For $6 \leq x < 7$:
  \begin{align*}
    P(X = 6) = F(6) - F(4) = \cfrac{7}{8} - \cfrac{3}{4} = \cfrac{1}{8}
  \end{align*}
  For $x \geq 7$:
  \begin{align*}
    P(X = 7) = 1 - F(6) = 1 - \cfrac{7}{8} = \cfrac{1}{8}
  \end{align*}
  Therefore, the probability mass function (PMF) of the random variable $X$ is:\\\\
  \[ P(X = x) = 
    \begin{cases} 
    \cfrac{1}{4} & \text{if } 1 \leq x < 3 \\\\
    \cfrac{3}{8} & \text{if } 3 \leq x < 4 \\\\
    \cfrac{1}{8} & \text{if } 4 \leq x < 7, x \geq 7 \\\\
    0 & \text{otherwise}
    \end{cases}
  \]

\newpage
\item 
  We want to compute the conditional probability that the gambler wins $i$ (where $i = 1, 2, 3$), given that he wins a positive amount. The conditional probability is given by:
  \[
  P(X = i | X > 0) = \cfrac{P(X=i\  \cap\  X>0)}{P(X > 0)}
  \]
  First, we calculate the probability of winning a positive amount:
  \[
  P(X > 0) = \cfrac{13}{55} + \cfrac{1}{11} + \cfrac{1}{165}
  \]
  Finally, we substitute these value into the conditional probability formula to obtain:
  \begin{align*}
    P(X = 1 | X > 0) = \cfrac{\cfrac{13}{55}}{\cfrac{13}{55} + \cfrac{1}{11} + \cfrac{1}{165}} = 0.7\overline{09}\\\\
  \end{align*}
  \[
    \boxed{\therefore P(X = 1 | X > 0) = 0.7\overline{09}}\\
  \]
  \begin{align*}
    P(X = 2 | X > 0) = \cfrac{\cfrac{1}{11}}{\cfrac{13}{55} + \cfrac{1}{11} + \cfrac{1}{165}} = 0.\overline{27}\\\\
  \end{align*}
  \[
     \boxed{\therefore P(X = 2 | X > 0) = 0.\overline{27}}\\
  \]
  \begin{align*}
    P(X = 3 | X > 0) = \cfrac{\cfrac{1}{165}}{\cfrac{13}{55} + \cfrac{1}{11} + \cfrac{1}{165}} = 0.0\overline{18}\\
  \end{align*}
  \[
     \boxed{\therefore P(X = 3 | X > 0) = 0.0\overline{18}}\\
  \]

\newpage
\item 
  \begin{enumerate}
    \item 
      A fair coin is tossed $n$ times, Let $X$ represent the difference between the number of Heads and the number of Tails obtained.\\
      Each toss of the coin can result in either a Head or a Tail. Therefore, the possible outcomes for $X$ depend on the number of Heads and Tails obtained.\\\\
      Considering the extreme cases:\\
      1. If there are \( n \) heads and 0 tails, \( X = n \) (all heads).\\
      2. If there are 0 heads and \( n \) tails, \( X = -n \) (all tails).\\\\
      Hence, the possible values of \( X \) range from \( -n \) to \( n \), inclusive. Therefore, the set of possible values for \( X \) is \( \{-n, -(n-2), -(n-4), \ldots, (n-4), (n-2), n\} \).\\
    \item 
    The probability mass function (PMF) describes the probability of each possible outcome,\\ For $n$ = 3:
    \begin{enumerate}
      \item \(P(X = -3)\): The probability of having 3 tails and 0 heads.
      \[ P(X = -3) = \left(\cfrac{1}{2^3}\right) = \cfrac{1}{8} \]
      \item \(P(X = -1)\): The probability of having 2 tails and 1 head or 1 tail and 2 heads.
      \[ P(X = -1) = \left(\cfrac{3}{2^3}\right) = \cfrac{3}{8} \]
      \item \(P(X = 1)\): The probability of having 2 heads and 1 tail or 1 head and 2 tails.
      \[ P(X = 1) = \left(\cfrac{3}{2^3}\right) = \cfrac{3}{8} \]
      \item \(P(X = 3)\): The probability of having 3 heads and 0 tails.
      \[ P(X = 3) = \left(\cfrac{1}{2^3}\right) = \cfrac{1}{8} \]
    \end{enumerate}
  \end{enumerate}

\newpage
\item 
  \textbf{We've been tasked with comparing poisson ratio and binomial Probability for some cases were $n$ is large but $p$ is very small, but their product \textit{i.e} $(\lambda = n\cdot p$) is a moderate number.}
  Here are the various cases to compare:
  \begin{enumerate}
    \item 
    \( P(X = 2) \) when \( n = 8, p = 0.1 \):\\
    Binomial Probability:
    \begin{align*}
      P(X = 2) &= \binom{8}{2} (0.1)^2 (0.9)^6 \\ 
      &= 0.1488
    \end{align*}   
    Poisson Approximation:
    \begin{align*}
      P(X = 2) &= \frac{e^{-0.8} \cdot (0.8)^2}{2!}\\
      &= 0.1438
    \end{align*}
    \item 
    \( P(X = 9) \) when \( n = 10, p = 0.95 \):\\
    Binomial Probability:
    \begin{align*}
      P(X = 9) &= \binom{10}{9} (0.95)^9 (0.05)^1 \\ 
      &= 0.3151
    \end{align*}   
    Poisson Approximation:
    \begin{align*}
      P(X = 9) &= \frac{e^{-9.5} \cdot (9.5)^9}{9!}\\
      &= 0.13
    \end{align*}
    \item 
    \( P(X = 0) \) when \( n = 10, p = 0.1 \):\\
    Binomial Probability:
    \begin{align*}
      P(X = 0) &= \binom{10}{0} (0.1)^0 (0.9)^{10} \\ 
      &= 0.3487
    \end{align*}   
    Poisson Approximation:
    \begin{align*}
      P(X = 0) &= \frac{e^{-1} \cdot (1)^0}{0!}\\
      &= 0.3679
    \end{align*}
    \item 
    \( P(X = 4) \) when \( n = 9, p = 0.2 \):\\
    Binomial Probability:
    \begin{align*}
      P(X = 4) &= \binom{9}{4} (0.2)^4 (0.8)^5 \\ 
      &= 0.066
    \end{align*}   
    Poisson Approximation:
    \begin{align*}
      P(X = 4) &= \frac{e^{-1.8} \cdot (1.8)^4}{4!}\\
      &= 0.0723
    \end{align*}
  \end{enumerate}
\end{enumerate}
\end{document}