\documentclass{article}
\newcounter{rownumbers}
\newcommand\rownumber{\stepcounter{rownumbers}\arabic{rownumbers}}
%\usepackage[a4paper, total={6in, 8in}]{geometry}
\usepackage{graphicx}
\usepackage{bm}
\usepackage{amsmath,amsfonts,mathtools}

\graphicspath{ {C:\Users\harsh\OneDrive\Documents} }
\usepackage{geometry}
 \geometry{
 a4paper,
 total={210mm,297mm},
 left=20mm,
 right=20mm,
 top=-2mm,
 bottom=2mm,
 }
 
%\usepackage[margin=0.5in]{geometry}

\usepackage{amsmath,amssymb}
\usepackage{ifpdf}
%\usepackage{cite}
\usepackage{algorithmic}
\usepackage{array}
\usepackage{mdwmath}
\usepackage{pdfpages}
\usepackage{mdwtab}
\usepackage{eqparbox}
\usepackage{parskip}
%\onecolumn
%\input{psfig}
\usepackage{color}
\usepackage{graphicx}
\setlength{\textheight}{23.5cm} \setlength{\topmargin}{-1.05cm}
\setlength{\textwidth}{6.5in} \setlength{\oddsidemargin}{-0.5cm}
\renewcommand{\baselinestretch}{1}
\pagenumbering{arabic}
\linespread{1.15}
\begin{document}
\textbf{
\begin{center}
{
\large{School of Engineering and Applied Science (SEAS), Ahmedabad University}\vspace{4mm}
}
\end{center}
%
\begin{center}
\large{Probability and Stochastic Processes (MAT277)\\ \vspace{4mm}
Homework Assignment-2\\\vspace{2mm}
Enrollment No: AU2140096 \hspace{4cm} Name: Ansh Virani }
\end{center}}
\vspace{2mm}

\vspace{10mm}

\begin{enumerate}
\item 
  \begin{enumerate}
    \item \textbf{Let event A be the probability that Omega is significant: $\Pr(A) = 0.75$}
    \item \textbf{Let there be an event that the Algorithm Delta produces a positive result, given that Omega is significant: $\Pr(B|A) = 1-0.15 = 0.85$}
    \item \textbf{Let there be an event that the Algorithm Delta produces a positive result, given that Omega is insignificant: $\Pr(B|A') = 0.15$}
    \item \textbf{The probability that Omega is insignificant: $\Pr(A') = 0.25$}\\
  \end{enumerate}
  \textit{The Probability that the message is significant given Algorithm Delta is positive is,}
  \begin{align*}
    Pr(A|B) &= \frac{Pr(B|A).Pr(A)}{Pr(B|A).Pr(A) + Pr(B|A').Pr(A')}\\\\
    &= \frac{(0.85). (0.75)}{(0.85). (0.75) + (0.15). (0.25)}\\\\
    &= \frac{0.6375}{0.675}\\\\
    &= 0.9444444
  \end{align*}
  \[
     \boxed{\therefore\ \Pr(A|B )\ \ \approx \ \ 0.945}\\
  \]

  \textit{Hence, as the probability that message contains critical information is less than 95\%, we should opt for further analysis.}\\\\
\newpage
\item \textbf{An undirected graph $G$ with $n$ nodes is given.}\\
  \textnormal{The min-cut algorithm works by iteratively contracting edges until only two nodes remain, representing the two disjoint sets of the minimum cut. The contraction process merges nodes, and the algorithm repeats until only two super-nodes are left.}\\
  \begin{enumerate}
    \item \textnormal{The total number of edges in the graph is given by the binomial coefficient \( \binom{n}{2} \), which represents all possible ways to choose 2 nodes out of n. This is equal to \(\cfrac{n(n-1)}{2}\).}\\\\
    \textnormal{During the contraction process, each contraction operation results in a unique cut. Since there are \(\cfrac{n(n-1)}{2}\) distinct edges in the graph, there can be at most \(\cfrac{n(n-1)}{2}\) distinct cuts formed during the contraction process.}\\
    \(\therefore\)  we can argue that there can be at most \(\cfrac{n(n-1)}{2}\) distinct min-cut sets in a graph $G$ with $n$ nodes using the analysis of the min-cut algorithm.\\
    \item 
  \end{enumerate}
\newpage
\item \textbf{Let A denote the event where number 6 is obtained while tossing a 6-sided unbiased dice.}\\
  \begin{enumerate}
    \item \textnormal{The probability of rolling a 6 on a single trial is: \(\Pr(A) = \cfrac{1}{6}\)}\\\\
    \textnormal{The probability of not rolling a 6 on a single trial is: \(\Pr(A') = \cfrac{5}{6}\)}\\\\
    Now, let us consider the event that a six occurs on the \(k^{th}\) roll. This implies that there were \(k-1\) consecutive non-six outcomes followed by a six. The probability of this sequence is given by:
    \begin{align*}
      \Pr(X= k) & = \Pr(A')^{(X-1)}.\Pr(A)\\\\
      \Pr(X= k)& =\ \left(\cfrac{5}{6}\right)^{k-1}\cdot \left(\cfrac{1}{6}\right)\\
    \end{align*}
    \item To Check the legitimacy of the PMF, we need to satisfy two conditions:
      \begin{enumerate}
        \item Each probability of P(X = k) is non-negative for all possible values of \(k\).
        \begin{align*}
          \Pr(X=k) \geq 0\\\\
          \left(\cfrac{5}{6}\right)^{k-1}\cdot \left(\cfrac{1}{6}\right) \geq 0
        \end{align*}
        This is ture because both the terms, \(\left(\cfrac{5}{6}\right)^{k-1}\) and \(\left(\cfrac{1}{6}\right)\) are non-negative.\\
        \item The sum of all probabilities over all possible values of \(k\) is equal to 1.
        \begin{align*}
          \sum_{k=1}^{n} P(X=k)\ =\ \sum_{k=1}^{n} \left(\cfrac{5}{6}\right)^{k-1}\cdot \left(\cfrac{1}{6}\right)
        \end{align*}
        It's a geometric series where the sum can be represented as:\\
        \begin{align*}
          \sum_{k=1}^{n} \left(\frac{5}{6}\right)^{k-1} \cdot \left(\frac{1}{6}\right) &= \left(\frac{1}{6}\right) \cdot \frac{1 - \left(\frac{5}{6}\right)^n}{1 - \frac{5}{6}} \\
          &= 1 - \left(\frac{5}{6}\right)^n
      \end{align*}
      This term approaches 1 as \(n\) tends to infinity, the sum of probabilities converges to 1.\\\\
      $\therefore$ this PMF is legitimate.
      \end{enumerate}
  \end{enumerate}
\end{enumerate}

\end{document}